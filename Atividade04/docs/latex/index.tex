\chapter{Atividade 04}
\hypertarget{index}{}\label{index}\index{Atividade 04@{Atividade 04}}
\label{index_md__2_users_2renanoliveira_2_desktop_2ufscar_22023-02_2computacao-grafica_2atividades-projeto_2_atividade04_2_r_e_a_d_m_e}%
\Hypertarget{index_md__2_users_2renanoliveira_2_desktop_2ufscar_22023-02_2computacao-grafica_2atividades-projeto_2_atividade04_2_r_e_a_d_m_e}%
 {\bfseries{Descrição\+:}}

Este projeto visa a implementação de uma técnica de renderização chamada traçado de raios (ray tracing), com base nos princípios apresentados nos tutoriais "{}\+Ray Tracing in One Weekend"{} e "{}\+Ray Tracing\+: The Next Week"{}, disponíveis em\+:

Tutorial 1\+: Ray Tracing in One Weekend Tutorial 2\+: Ray Tracing\+: The Next Week A atividade se concentra na visualização de uma esfera e de um triângulo, integrando conceitos desenvolvidos nas Atividades 1 e 2. Além disso, incorpora a implementação da Atividade 3, que permite visualizar um objeto complexo lido a partir de um arquivo.

Funcionalidades Principais\+:

Esfera e Triângulo\+: Implementação da visualização de esferas e triângulos utilizando técnicas de traçado de raios.

Classes de Manipulação de Vetores\+: Utilização das classes desenvolvidas nas Atividades 1 e 2 para manipulação eficiente de vetores.

Leitura de Objeto a partir de Arquivo\+: Implementação da leitura de objetos a partir de arquivos no formato .obj, permitindo a visualização de modelos 3D mais complexos.

{\bfseries{Exemplo de Uso\+:}}


\begin{DoxyItemize}
\item adicionar Lib para salvar em PNG 
\begin{DoxyCode}{0}
\DoxyCodeLine{\$\ git\ clone\ https://github.com/lvandeve/lodepng.git}

\end{DoxyCode}

\item compilar e executar o código, as imagens serão salvas na pasta {\ttfamily /outputs} 
\begin{DoxyCode}{0}
\DoxyCodeLine{\$\ g++\ -\/o\ output\ main.cpp}
\DoxyCodeLine{\$\ ./output}

\end{DoxyCode}
 
\end{DoxyItemize}